\chapter{Przykładowy rozdział}

The general formatting requirements for the diploma thesis are listed below:
\begin{itemize}
	\item sheet size: A4,
	\item paper orientation: vertical,
	\item font: Arial,
	\item basic font size: 10 pt.,
	\item line spacing: 1.5,
	\item mirror margins:
	\begin{itemize}
		\item top: 2.5 cm,
		\item bottom: 2.5 cm,
		\item internal: 3.5 cm,
		\item external: 2.5 cm,
	\end{itemize}
	\item the thesis text should be justified (aligned to both margins),
	\item each paragraph should begin with a 1.25 cm indentation.
\end{itemize}

The thesis should be prepared for double-sided printing. The page numbering should be in the page footer and centred. The title page should include the author’s (authors’) Statement (Statements) and the page number should not be printed. Page numbering, should be in Arabic numerals using a 9 pt. Arial font. It should begin on page 3 (the Table of Contents) and continue to the last page.

An example of the correct way of presenting information in points (bulleted list) is shown above. Each point (line of text) should be preceded by a bullet. It should begin in the lower case and end with a comma or semi-colon, except for the last point (line of text), which should end with a full stop.

The title of a table should be directly above the table, with a 9 pt. font size and not ended with a full stop, as shown above. Paragraph spacing for the text in the table is as follows:
\begin{itemize}
	\item top 6 pt.,
	\item bottom 0 pt.
\end{itemize}

The correct headings are presented in Table~\ref{tab:heading-styles}.

\begin{table}[h]
	\caption{Sizes and styles of headings}
	\label{tab:heading-styles}
	\begin{tabularx}{\textwidth}{|X|X|X|}
		\hline
		Level of heading	& Example 					& Font size and style \\ \hline
		Heading 1 			& \textbf{1. CHAPTER TITLE}			& 12 pt., CAPITALS, bold \\ \hline
		Heading 2			& \textbf{\textit{1.1. Subchapter title}}		& 10 pt., bold and in italics \\ \hline
		Heading 3			& \textit{1.1.1. Subchapter section}	& 10 pt., italics \\ \hline
	\end{tabularx}
\end{table}

Data should be presented in the table as in Table~\ref{tab:heading-styles}, shown above, i.e. using a 9 pt. font and aligning text to the left edge of the cell.

Table numbering is continuous within the chapter. The table sequence number (table title) is preceded by the word Table and the number of the chapter, ended with a dot (e.g. Table~1.1. Size\ldots). Every table must be referred to in the thesis text, e.g.`Table~\ref{tab:heading-styles}. contains\ldots'.

If a table needs to be continued on more than one page, the table heading should appear on each subsequent page, using the option: TABLE PROPERTIES -> row -> repeat as heading row at the top of every page.

The first paragraph below a table should begin with a top margin of 12 pt.

Lines of text should not end with short prepositions, such as: a, an, the, in, on, etc. In such cases the non-breaking space (NBSP), using ctrl, shift and space, is recommended instead of an ordinary space.
