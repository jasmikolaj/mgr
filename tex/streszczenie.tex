\chapter*{Streszczenie}

Celem niniejszej pracy magisterskiej było stworzenie narzędzia, służącego do projektowania stron internetowych zawierających wizualizacje pracy urządzeń energetyki kolejowej. Narzędzie to zostało zaimplementowane jako aplikacja internetowa o nazwie Elements. Praca jest podzielona na pięć rozdziałów.

Pierwszy z nich zawiera opis systemu sterowania urządzeniami elektrycznego ogrzewania rozjazdów kolejowych. System zorganizowany jest w strukturę, w skład której wchodzą określone warstwy. Są nimi: warstwa urządzeń wykonawczych, autonomicznych rozdzielnic energetycznych, archiwizacji i obróbki danych, nadzoru i zarządzania oraz komunikacji. Każda z nich jest dokładnie opisana w odpowiednim podrozdziale. Szczególną uwagę poświęcono hierarchii urządzeń oraz komunikacji między nimi, gdyż miały one duży wpływ podczas tworzenia aplikacji.

Rozdział drugi opisuje proces wyboru technologii, za pomocą których została stworzona aplikacja. Wybór technologii back-endu był narzucony przez istniejące biblioteki wykonane w języku C\#, których użycie było konieczne do zrealizowania projektu. Wybór narzędzi służących do wykonania części front-end był trudniejszy z uwagi na dużą ilość istniejących rozwiązań do porównania. Rozdział zawiera ich analizę, porównanie oraz wyjaśnienie przyczyn wyboru jednej z nich. Wybrane technologie zostały szczegółowo opisane w rozdziale trzecim.

W rozdziale czwartym wyjaśniono w jaki sposób rozwiązanie zostało zaprojektowane. Aplikacja została nazwana Elements i podzielona na dwa moduły ze względu na ich odrębne funkcje. Pierwszy z nich - Designer służy do projektowania widoków z komponentów. Drugi, o nazwie Viewer, prezentuje wcześniej stworzone projekty użytkownikowi końcowemu.

Ostatni fragment pracy opisuje metody zastosowane do implementacji aplikacji. Opisane zostały jej składowe: warstwa serwera aplikacji jak i interfejsu użytkownika. Rozdział zawiera opis implementacji dwóch modułów, na które została podzielona aplikacja. Porusza również kwestię elementów reprezentujących zmienne urządzeń oraz mechanizm pobierania ich i prezentacji na stronie internetowej.
