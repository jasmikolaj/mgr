\chapter*{Streszczenie}

Niniejsza praca magisterska miała na celu zaprojektowanie oraz stworzenie aplikacji służącej do projektowania wizualizacji pracy urządzeń energetyki kolejowej. Głównym celem aplikacji było zaprezentowanie nowej metody prezentacji pracy sterowników, polegającej na dostarczaniu danych w czasie rzeczywistym. Czynności projektowania wizualizacji, jak i prezentacji projektów miały być dostępne z poziomu przeglądarki internetowej.
Rozwiązanie zostało stworzone w formie aplikacji internetowej oraz pomyślnie przeszło testy jako jeden z wielu elementów systemu sterowania urządzeniami firmy \textit{Zakład Automatyki i Urządzeń Pomiarowych Arex sp. z o. o.} Projekt miał na celu zaprezentowanie nowego sposobu zdalnego monitoringu urządzeń. Obecnie aplikacja nie ma zastosowania komercyjnego.