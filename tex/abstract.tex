\chapter*{Abstract}

This paper relates to the subject of railway heating devices monitoring. The goal of this work was to investigate and create new way of devices monitoring based on data delivery in real time.

Railway heating devices are organised in tree structure and divided into two groups. The heating drivers control the heaters and send measurements to the controller devices called NEK. NEKs gather data from all devices placed on whole station and push them further to the data servers. This gave potential opportunity to gather the data from the server in real time. This function became the first main requirement of the project. Heating drivers are of many types and that is why the second function of the project was to bring functionality of creating devices work visualisation projects with no detailed knowledge about devies specification. Application was meant to preserve high availability therefore it was created as web application with rich user interface.


The final product was deployed as a part of the Polish Railway Heating Monitoring System. The main goal of deploy was to deliver proof of concept of devices monitoring in real time. At this point, project is not used in commercialy.


\vspace{12pt}
\noindent\textbf{Keywords:} railway heating, devices monitoring, scada, rich interface, real time data delivery

\vspace{12pt}
\noindent\textbf{Field of science and technology in accordance with OECD requirements}: engineering and technology science – computer science